\documentclass[14pt]{extarticle}
\usepackage{extsizes}
\usepackage{geometry}
\usepackage{xcolor}
\geometry{margin=0.5in}

%% for images
\usepackage{graphicx}
\graphicspath{ {images/} }

%% language support
\usepackage[T1,T2A]{fontenc}
\usepackage[utf8]{inputenc}
\usepackage[english,russian]{babel}

\usepackage{amsmath}
\usepackage{tikz}

%% hyperrefs
\usepackage{hyperref}
\hypersetup{
    colorlinks,
    citecolor=black,
    filecolor=black,
    linkcolor=black,
    urlcolor=black
}

\title{Блядское бдз v15 сука ненавижу тфкп}
\author{Я нахуй, а кто ещё}
\date{28.03.2024}

\begin{document}
\maketitle
\section{Найти $|z|, Re(z), Im(z)$}
1) 
\begin{displaymath}
    z = {\Bigg(\frac{-3-i\sqrt3}{1-i}\Bigg)}^{96}=(z^*)^{96} 
\end{displaymath}
\begin{displaymath}
    z^* = \frac{-3-i\sqrt3}{1-i} = \frac{3+i\sqrt3}{i-1}=
    \frac{(3+i\sqrt3)(i+1)}{(i-1)(i+1)} = \frac{3i-\sqrt3+3+i\sqrt3}{-1-1} =
\end{displaymath}
\begin{displaymath}
    =\frac{3-\sqrt3}{-2}+i \ \frac{3+\sqrt3}{-2}=\frac{-3-\sqrt3}{2}+
    i \ \Bigg(-\frac{3+\sqrt3}{2}\Bigg)
\end{displaymath}
\begin{displaymath}
    |z^*| = \sqrt{\Bigg(-\frac{3-\sqrt3}{2}\Bigg)^2
    +\Bigg(-\frac{3+\sqrt3}{2}\Bigg)^2}=
    \sqrt{\frac{9-6\sqrt3+3}{4}+\frac{9+6\sqrt3+3}{4}}=\sqrt{\frac{24}{4}}
    =\sqrt{6}
\end{displaymath}
\begin{displaymath}
    z^*=\sqrt6\Bigg(-\frac{3-\sqrt3}{2\sqrt6}
    +i\Bigg(-\frac{3+\sqrt3}{2\sqrt6}\Bigg)\Bigg)
\end{displaymath}
\begin{displaymath}
    \cos\phi = -\frac{3-\sqrt3}{2\sqrt6}, \ 
    \sin\phi = -\frac{3+\sqrt3}{2\sqrt6}
\end{displaymath}
\begin{displaymath}
    \tg\phi = \frac{\sin\phi}{\cos\phi}=\frac{3+\sqrt3}{3-\sqrt3}
    =\frac{\sqrt3+1}{\sqrt3-1}=\frac{(\sqrt3+1)^2}{3-1}
    =\frac{4+2\sqrt{3}}{2}=2+\sqrt3
\end{displaymath}
\begin{displaymath}
    -\frac{\pi}{2} < \arctan \phi < \frac{\pi}{2}
\end{displaymath}
\begin{displaymath}
    \phi=\arctan{(\tan\phi)}-\pi=\arctan{(2+\sqrt3)}-\pi
\end{displaymath}
\begin{displaymath}
    z^*=\sqrt6 \ [\ \cos \ (\arctan{(2+\sqrt3)}-\pi)+ \ 
    i \ \sin \ (\arctan{(2+\sqrt3)}-\pi) \ ]
\end{displaymath}\\
\begin{displaymath}
    \color{magenta}|z|=(z^*)^{96}=(\sqrt6)^{96}=6^{48}
\end{displaymath}
\begin{displaymath}
    z=(z^*)^{96}= \textcolor{olive}{6^{48} * \cos(96[\arctan{(2+\sqrt3)}-\pi])}
    + \ i * \textcolor{teal}{6^{48} * \sin(96[\arctan{(2+\sqrt3)}-\pi])}
\end{displaymath}
\begin{displaymath}
    = \textcolor{olive}{Re(z)} + \textcolor{teal}{Im(z)}
\end{displaymath}
2)
\begin{displaymath}
    z=\sqrt[3]{8-8i}\rightarrow z^3=z^*=8-8i
\end{displaymath}
\begin{displaymath}
    |z^*|=\sqrt{8^2+8^2}=\sqrt{2*8^2}=\sqrt{2}*8=8\sqrt{2}
\end{displaymath}
\begin{displaymath}
    z^*=8\sqrt{2}\Bigg(\frac{1}{\sqrt{2}}+ \ i (-\frac{1}{\sqrt{2}}) \Bigg)
\end{displaymath}
\begin{displaymath}
    \cos\phi = \frac{1}{\sqrt{2}}, \ 
    \sin\phi =-\frac{1}{\sqrt{2}}\rightarrow\phi=-\frac{\pi}{4}
\end{displaymath}
\begin{displaymath}
    z^*=8\sqrt{2} \ (\ \cos(-\frac{\pi}{4})+ \ 
    i \ \sin(-\frac{\pi}{4}) \ )
\end{displaymath}
\begin{displaymath}
    \color{magenta}|z|=\sqrt[3]{|z^*|}=\sqrt[3]{8\sqrt{2}}=2\sqrt[6]{2}
\end{displaymath}
\begin{displaymath}
    z=\sqrt[3]{z^*}=
    2\sqrt[6]{2}\Bigg[\cos\Biggl(\frac{-\frac{\pi}{4}+2\pi k}{n}\Biggr)\Bigg]
    + i \ \Biggl[\sin\Biggl(\frac{-\frac{\pi}{4}+2\pi k}{n}\Biggr)\Biggr]
\end{displaymath}
\begin{displaymath}
    n=3 \rightarrow k \in \{0;1;2\}
\end{displaymath}
$k=0:$
\begin{displaymath}
    z_0=2\sqrt[6]{2} \ \Biggl(\cos \frac{-\frac{\pi}{4}+2\pi * 0}{3}+
    i \ * \sin \frac{-\frac{\pi}{4}+2\pi * 0}{3}\Biggr)=
\end{displaymath}
\begin{displaymath}
    = \textcolor{olive}{2\sqrt[6]{2} \ \cos (-\frac{\pi}{12})}+
    i * \textcolor{teal}{2\sqrt[6]{2}* \sin (-\frac{\pi}{12})}
\end{displaymath}
$k=1:$
\begin{displaymath}
    z_0=2\sqrt[6]{2} \ \Biggl(\cos \frac{-\frac{\pi}{4}+2\pi * 1}{3}+
    i \ * \sin \frac{-\frac{\pi}{4}+2\pi * 1}{3}\Biggr)=
\end{displaymath}
\begin{displaymath}
    = \textcolor{olive}{2\sqrt[6]{2} \ \cos \Biggl(\frac{7\pi}{12}\Biggr)}+
    i * \textcolor{teal}{2\sqrt[6]{2}* \sin \Biggl(\frac{7\pi}{12}\Biggr)}
\end{displaymath}
$k=2:$
\begin{displaymath}
    z_0=2\sqrt[6]{2} \ \Biggl(\cos \frac{-\frac{\pi}{4}+2\pi * 2}{3}+
    i \ * \sin \frac{-\frac{\pi}{4}+2\pi * 2}{3}\Biggr)=
\end{displaymath}
\begin{displaymath}
    = \textcolor{olive}{2\sqrt[6]{2} \ \cos \Biggl(\frac{15\pi}{12}\Biggr)}+
    i * \textcolor{teal}{2\sqrt[6]{2}* \sin \Biggl(\frac{15\pi}{12}\Biggr)}
\end{displaymath}
3)
Ебучая пизда с гиперболическими функциями. Важные формулы:
\begin{displaymath}
    \cosh z = \frac{e^z+e^{-z}}{2} \rightarrow 
    \cosh^2 z = \frac{e^{2z}+e^{-2z}+2}{4}
\end{displaymath}
\begin{displaymath}
    \sinh z = \frac{e^z-e^{-z}}{2} \rightarrow 
    \sinh^2 z = \frac{e^{2z}+e^{-2z}-2}{4}
\end{displaymath}
$$\cosh ^2 z - \sinh ^2 z = 1$$
\\
\begin{displaymath}
    \cos z =\frac{e^{iz}+e^{-iz}}{2} \rightarrow 
    \cos^2 z = \frac{e^{2iz}+e^{-2iz}+2}{4}
\end{displaymath}
\begin{displaymath}
    \sin z = \frac{e^{iz}-e^{-iz}}{2i} \rightarrow 
    \sin^2 z = -\frac{e^{2iz}+e^{-2iz}-2}{4}
\end{displaymath}
$$\cos ^2 z + \sin ^2 z = 1$$
\\
$\cosh iz = \cos z = \cos (-z)\\ -i \sinh (-z) = i\sinh z = \sin iz\\
2\cosh^2z - 1 = 2\frac{e^{2z}+2+e^{-2z}}{4} - \frac{2}{2}
= \frac{e^{2z}+e^{-2z}}{2}=\cosh2z$
\begin{displaymath}
    z=\tan(2+i)=\frac{\sin(2+i)}{\cos(2+i)}
\end{displaymath}
\begin{displaymath}
    \sin(2+i) = \sin2\cos i + \sin i \cos 2 = 
    \sin 2 \cosh 1 + i \sinh 1 \cos 2
\end{displaymath}
\begin{displaymath}
    \cos(2+i) = \cos2\cos i - \sin2 \sin i = 
    \cos 2 \cosh 1 - i \sin2 \sinh 1
\end{displaymath}
\begin{displaymath}
    z= \frac{x_1x_2+y_1y_2}{x_2^2+y_2^2} + i \frac{x_2y_1-x_1y_2}{x_2^2+y_2^2}
\end{displaymath}
\begin{displaymath}
    = \frac{\sin 2 \cosh 1 \cos 2 \cosh 1 + \sinh 1 \cos 2 (-\sin2 \sinh 1)}
    {\cos^2 2 \cosh^2 1 + \sin^2 2 \sinh^2 1}
\end{displaymath}
\begin{displaymath}
    + i \ \frac{\cos 2 \cosh 1 \sinh 1 \cos 2 - \sin 2 \cosh 1 (-\sin2 \sinh 1)}
    {\cos^2 2 \cosh^2 1 + \sin^2 2 \sinh^2 1}
\end{displaymath}
\begin{displaymath}
    = \frac{\sin 2 \cos 2 (\cosh^2 1 - \sinh^2 1)}
    {\cos^2 2 \cosh^2 1 + (1 - \cos^2 2) \sinh^2 1}
\end{displaymath}
\begin{displaymath}
    + i \ \frac{\cosh 1 \sinh 1(\cos^2 2 + \sin^2 2)}
    {\cos^2 2 \cosh^2 1 + (1 - \cos^2 2) \sinh^2 1}
\end{displaymath}
\begin{displaymath}
    = \frac{\sin 2 \cos 2}
    {\cos^2 2 (\cosh^2 1 - \sinh^2 1)+\sinh^2 1}
\end{displaymath}
\begin{displaymath}
    + i \ \frac{\cosh 1 \sinh 1}
    {\cos^2 2 (\cosh^2 1 - \sinh^2 1)+\sinh^2 1}
\end{displaymath}
\begin{displaymath}
    = \frac{\sin 2 \cos 2 + i \ \cosh 1 \sinh 1}
    {\cos^2 2 +\sinh^2 1}
\end{displaymath}
\begin{displaymath}
    = \frac{\sin 4 + i \ \sinh 2}
    {2\cos^2 2 +2(\cosh^2 1-1)}
\end{displaymath}
\begin{displaymath}
    = \frac{\sin 4 + i \ \sinh 2}
    {2\cos^2 2 -1 +2\cosh^2 1-1}
\end{displaymath}
\begin{displaymath}
    = \frac{\sin 4 + i \ \sinh 2}
    {\cos4 + \cosh 2}
\end{displaymath}
\begin{displaymath}
    = \textcolor{olive}{\frac{\sin 4}{\cos4 + \cosh 2}}
    +i \ \textcolor{teal}{\frac{ \sinh 2}{\cos4 + \cosh 2}}
\end{displaymath}
\begin{displaymath}
    \color{magenta}|z|=\sqrt{\frac{\sin^2 4+\sinh^2 2}{(\cos4 + \cosh 2)^2}}=
    \frac{\sqrt{\sin^2 4+\sinh^2 2}}{|\cos4 + \cosh 2|}
\end{displaymath}
\\ \\
4)
\begin{displaymath}
    z=\ln\biggl(\frac{1+i}{1-i\sqrt{3}}\biggr)=\ln z^*
\end{displaymath}
\begin{displaymath}
    z^* = \frac{(1+i)(1+i\sqrt{3})}{4}= \frac{(1-\sqrt{3})+i \ (1+\sqrt{3})}{4}
\end{displaymath}
\begin{displaymath}
    |z^*|=\sqrt{\frac{(1-\sqrt{3})^2+(1+\sqrt{3})^2}{4^2}}=\frac{1}{\sqrt{2}}
\end{displaymath}
\begin{displaymath}
    z^*=\frac{1}{\sqrt{2}}
    \biggl(\frac{1-\sqrt{3}}{2\sqrt{2}}+i \ \frac{1+\sqrt{3}}{2\sqrt{2}}\ \biggr)
\end{displaymath}
\begin{displaymath}
    \tan \phi = -\frac{1+\sqrt{3}}{1-\sqrt{3}}=-\frac{(1+\sqrt{3})^2}{-2}=
    \frac{4+2\sqrt{3}}{2}=2+\sqrt{3}, 
\end{displaymath}
\begin{displaymath}
    \phi = \pi - \arctan(\tan \phi)=\pi - \arctan(2+\sqrt{3})
\end{displaymath}
\begin{displaymath}
    z^*=\frac{1}{\sqrt{2}}
    \biggl[\cos 
    \Biggl(\pi - \arctan\Biggl(2+\sqrt{3}\Biggr)\Biggr)
    +i \ \sin \ \Biggl(\pi - \arctan\Biggl(2+\sqrt{3}\Biggr)\Biggr)\ \biggr]
\end{displaymath}
\\ \\
$\ln z = \ln (\rho * e ^ {i\phi}) = \ln \rho + i \phi$\\
$\ln \frac{1}{\sqrt{2}} = \ln (2^{-\frac{1}{2}}) = -\frac{1}{2}\ln 2$
\begin{displaymath}
    z=\ln z^* = \textcolor{olive}{-\frac{1}{2}\ln 2}
    +i \ \textcolor{teal}
    {\Biggl(\pi - \arctan\Biggl(2+\sqrt{3}\Biggr)\Biggr)}
\end{displaymath}
\begin{displaymath}
    \color{magenta}|z|=\sqrt{\frac{1}{4}\ln ^2 2+
    \Biggl(\pi - \arctan\Biggl(2+\sqrt{3}\Biggr)\Biggr)^2}
\end{displaymath}
Замечание: эту конскую ебанину можно упростить, если решать другим способом. 
Напиши, если тебя это интересует.\\ 
\\
5)
\begin{displaymath}
    z=(1+ i \ \sqrt{3})^{1-2i}=e^{(Ln(1+ i \ \sqrt{3}))*(1-2i)}
\end{displaymath}
\begin{displaymath}
    Ln(1+ i \ \sqrt{3})
    =Ln \Bigl(2\Bigl(\frac{1}{2}+i\ \frac{\sqrt{3}}{2}\Bigr)\Bigr)=
    \ln2 + i\ (\frac{\pi}{3}+2\pi k)
\end{displaymath}
\begin{displaymath}
    z=e^{(\ln2 + i\ (\frac{\pi}{3}+2\pi k))(1-2i)}
    =e^{\ln2+2(\frac{\pi}{3}+2\pi k)+i((\frac{\pi}{3}+2\pi k)-2\ln2)}=
\end{displaymath}
\begin{displaymath}
    \textcolor{olive}
    {2e^{\frac{2\pi}{3}+4\pi k}(\cos (\frac{\pi}{3}+2\pi k-\ln4))}+
    i\ \textcolor{teal}
    {2e^{\frac{2\pi}{3}+4\pi k}(\sin (\frac{\pi}{3}+2\pi k-\ln4))}
    , \color{magenta}|z| = 2e^{\frac{2\pi}{3}+4\pi k}
\end{displaymath}
\begin{displaymath}
\end{displaymath}
\begin{displaymath}
\end{displaymath}
\begin{displaymath}
\end{displaymath}
\end{document}
